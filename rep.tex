% !Mode:: "TeX:UTF-8"
\documentclass[compress,mathserif,red]{beamer}
\usepackage{indentfirst,fancyhdr,graphicx}
\usepackage[SlantFont,BoldFont]{xeCJK}
\usepackage{mathrsfs}
\usepackage{amsmath}
\usepackage{amsfonts}
\usepackage{amssymb}
\usepackage{amsthm}
\usepackage{enumerate}
\usetheme{Antibes} % Beamer theme v 3.0
\usecolortheme{lily}
\beamertemplateshadingbackground{white!15}{yellow!50}

\setCJKmainfont[BoldFont={Adobe Heiti Std}, ItalicFont={Adobe Kaiti Std}]{Adobe Song Std}
\setCJKsansfont{Adobe Heiti Std}
\setCJKmonofont{Adobe Fangsong Std}
\setCJKfamilyfont{Microsoft YaHei}{Microsoft YaHei}
\setCJKfamilyfont{Adobe Song Std}{Adobe Song Std}
\setCJKfamilyfont{Adobe Heiti Std}{Adobe Heiti Std}
\setCJKfamilyfont{Adobe Kaiti Std}{Adobe Kaiti Std}
\setCJKfamilyfont{Adobe Fangsong Std}{Adobe Fangsong Std}
\setmainfont{Times New Roman}
\setsansfont[BoldFont={Courier New Bold}]{Courier New}
\setmonofont[BoldFont={Arial:style=Bold}]{Arial}
\begin{document}


\title{薛定谔算子的谱}
\author{赵鑫}
\institute{中国科学技术大学数学系}
\date{\today}



\frame{\titlepage}





\begin{frame}
\frametitle{薛定谔算子的谱}
在这篇论文中主要介绍了关于薛定谔的谱。首先引出了连续泛函的微积分。
然后介绍了谱测度,利用谱测度证明了自伴算子的谱理论的主要结果--积算子形式下的谱定理。
运用Radon–Nikodym定理将谱分解。介绍了两个关于薛定谔算子的重要结果.
最后对此谱定理进行了应用,得到了在一个较一般的情况下,存在一个剩余集使得遍历薛定谔算子没有绝对连续谱.
\end{frame}
\section{Borel functional calculus}
\subsection{polynomial on bounded self-adjoint operators}
\begin{frame}
\frametitle{谱}
\begin{corollary}
设$P(x)=\sum_{n=0}^N {a_n
x^n}$和$P(A)=\sum_{n=0}^N {a_n
 A^n}$. 那么
 $$\sigma(P(A))=\{P(\lambda) \mid \lambda \in
 \sigma(A)\}$$.
\end{corollary}
\end{frame}



\subsection{polynomial on bounded self-adjoint operators}
\begin{frame}
\frametitle{模}
\begin{corollary}
令$A$为一个有界自伴算子。那么$$\| P(A)
 \|=\sup_{\lambda \in \sigma(A)}\left| P(\lambda) \right|.$$
\end{corollary}
\end{frame}




\subsection{continuous functions on bounded self-adjoint operators}
\begin{frame}
\frametitle{$C(\sigma)$作用在有界自伴算子上}
\begin{theorem}
设$A$是$Hilbert$空间$\mathcal{H}$上的有界自伴算子.那么存在唯一的映射$\phi
 :C(\sigma (A))\rightarrow \mathcal{L}(\mathcal{H})$
\end{theorem}
\end{frame}




\section{spectrum measure}
\subsection{definition}
\begin{frame}
\frametitle{引子}
令$\varphi \in \mathcal{H}$. 那么$f\rightarrow(\varphi,f(A)
 \varphi)$是在$C(\sigma(A))$上的正线性泛函. 因此由Riesz-Markov
 定理, 在紧集$\sigma(A)$上存在唯一的测度$\mu_{\varphi}$满足
 $$(\varphi,f(A)
 \varphi)=\int_{\sigma(A)} f(\lambda)d\mu_{\varphi},\forall f\in C(\sigma(A))$$.

\end{frame}




\begin{frame}
\frametitle{定义}
\begin{definition}
B测度$\mu_{\varphi}$叫做向量$\varphi$的谱测度
\end{definition}

\end{frame}


\subsection{spectral theory}
\begin{frame}
\frametitle{定理}
\begin{definition}
一个向量$\varphi \in
\mathcal{H}$叫做$A$的循环向量, 如果有限的$\{A^n \varphi
\}_{n=0}^{\infty}$的线性组合在$\mathcal{H}$中稠密.
\end{definition}

\medskip
\begin{theorem}
设$A$是$\mathcal{H}$上一个有界自伴算子。如果$A$具有循环向量是$\varphi$.
那么存在酉算子$U: \mathcal{H} \rightarrow L^2(\sigma(A),d\mu_{\varphi})$
满足$$UAU^{-1}f(\lambda)=\lambda
f(\lambda),$$ 其中$f \in L^2(\sigma(A),d\mu_{\varphi})$.

\end{theorem}

\end{frame}

\begin{frame}
\frametitle{谱定理}
\begin{theorem}
设$A$是可分Hilbert空间$\mathcal{H}$上的一个有界自伴算子.那么,在$\sigma(A)$上存在测度$\{\mu_n\}_{n=1}^N,
 N\in
\{1,2,\ldots,\}\cup \{+\infty\}$和酉算子$U: \mathcal{H} \rightarrow
\bigoplus _{n=1}^N
L^2(\mathbb{R},d\mu_n)$满足$$(UAU^{-1}\varphi_n)(\lambda)=\lambda
\varphi_n(\lambda)$$将$\varphi \in\bigoplus _{n=1}^N
L^2(\mathbb{R},d\mu_n)$写作$N$元组$\langle
\varphi_1(\lambda),\ldots,\varphi_N(\lambda) \rangle$.
这种对$A$的表示就叫做谱表示.
\end{theorem}

\end{frame}


\section{Spectral decomposition} % Bookmark information
\subsection{Radon-Nikodym theorem} % Bookmark information
\begin{frame}
\frametitle{Radon-Nikodym定理}
\begin{definition}
如果$\mu$ 和 $\nu$ 是两个在可测空间$(\Omega,\Sigma)$上$\sigma$-有限的测度,那么就存在两个$\sigma$-有限的测度$\nu_0$和
$\nu_1$满足:

 (a) $\nu_0(E)=\int f d\mu.$

 (b) $\nu_1$和$\mu$ 是奇异的.

并且这两个测度是唯一确定的.
\end{definition}
\end{frame}


\begin{frame}
\frametitle{refinement}
\begin{corollary}
$\mathbb{R}$上的一个测度$\nu$可以被分解
$$\nu=\nu_{ac}+\nu_{sc}+\nu_{pp}.$$
其中$\Sigma_{ac}$为谱中相对于$R$上的Lebesgue测度绝对连续的部分;\\
$\Sigma_{sc}$为谱中奇异部分的连续的部分;\\
$\Sigma_{pp}$为纯点测度.
\end{corollary}
\end{frame}



\subsection{spectral decomposition}
\begin{frame}
\frametitle{谱分解}
\begin{theorem}
$$\sigma(T)=\sigma_{ac}(T)+\sigma_{sc}(T)+\sigma_{pp}(T).$$
\end{theorem}
\end{frame}


\section{Spectral analysis of ergodic Schrodinger operator} % Bookmark information
\subsection{definition} % Bookmark information
\begin{frame}
\frametitle{定义}
\begin{definition}
$T:\Omega\rightarrow\Omega$是可逆的,如果对任何满足
$$\mu(T^{-1}(E)\triangle(E))=0$$
的$E \in \Omega$
有$$\mu(E)=0 \  or \  \mu(E)=1$$
则称$T$是遍历的.
\end{definition}
\begin{definition}
$T:\Omega\rightarrow\Omega$是可逆遍历的,$f:\Omega \rightarrow \mathbb{R}$是有界可测函数,
$$V_\omega(n)=f(T^n \omega),\omega \in \Omega, n \in \mathbb{Z}$$
$V_\omega$称为位势.
\end{definition}

\end{frame}

\begin{frame}
\frametitle{遍历薛定谔算子}
\begin{definition}
$$[H_\omega \Psi](n)=\Psi(n+1)+\Psi(n-1)+V_\omega(n)\Psi(n)$$
其中$V_\omega$是位势.
则${ \{ H_\omega \} }_{\omega \in \Omega}$称为薛定谔算子的遍历族.
\end{definition}
\end{frame}

\subsection{main theorem} % Bookmark information
\begin{frame}
\frametitle{主要定理}
\begin{theorem}
存在$\Sigma \in \mathbb{R}$使得对于$\mu$几乎所有的$\omega$有
$$\sigma(H_\omega)=\Sigma$$
\end{theorem}

\medskip
L. Pastur,Spectral properties of disordered systems in the one-body approximation,
Communications in Mathematical Physics, 1980 - Springer
\end{frame}

\begin{frame}
\frametitle{主要定理}
\begin{theorem}
存在$\Sigma_\bullet \in \mathbb{R}$使得对于$\mu$几乎所有的$\omega$有
$$\sigma_\bullet(H_\omega)=\Sigma_\bullet,\bullet \in { \{ac,sc,pp \} }$$
\end{theorem}

\medskip
H. Kunz and B. Souillard, Sur le spectre des op\'erateurs aux différences finies al\'eatoires,
Comm. Math. Phys. 78 (1980/81), no. 2, 201-246.
\end{frame}



\section{generic singular spectrum for ergodic Schrodinger operators} % Bookmark information
\subsection{the absolute continuous spectrum} % Bookmark information


\begin{frame}
\frametitle{绝对连续谱的确定}
\begin{theorem} (Ishii 1973, Pastur 1980,Kotani 1984)
$$\Sigma_{ac}={\overline{ \{ E \in \mathbb{R}:\gamma(E)=0 \} }}^{ess}$$
\end{theorem}

\end{frame}

\begin{frame}
\frametitle{绝对连续谱的确定}

$$M(f)=|\{E \in \mathbb{R} :\gamma_f(E)=0 \} |$$
其中$| \cdot |$是Lebesgue测度.由定理八,我们知道
$$\sigma(f)=\phi$$
iff
$$M(f)=0$$

\end{frame}


\subsection{Lyapunov exponent and m-function} % Bookmark information
\begin{frame}
\frametitle{Lyapunov指数}
$$\gamma_f(E)=\lim_{n\to\infty}\int_\Omega \ln \| S^n_{f,E} (\omega) \|$$
其中
$$S^n_{f,E} (\omega)=S_{f,E}(T^{n-1}(\omega)) \cdot \cdot \cdot S_{f,E}(\omega)$$
$$S_{f,E}(\omega)=\left( \begin{array}{ccc}
E-f(\omega) & -1  \\

1 & 0  \end{array} \right) $$
\end{frame}

\begin{frame}
\frametitle{$m$ -函数}
若$Im E>0$,有
$$\gamma_f(E)=\int_\Omega -Re \ln m_{\omega,f}(E)d\mu(\omega)$$
其中
$$ m_{\omega,f}(E)=\lim_{n\to\infty}S_{f,E}(T^{n}(\omega))\cdot i $$


\end{frame}





\subsection{generic singular spectrum for ergodic Schrodinger operators} % Bookmark information
\begin{frame}
\frametitle{几个引理}
\begin{lemma}
对于$r>0,\Lambda>0$,函数
$$(L^1(\Omega)\bigcap B_r(L^{\infty}(\Omega)),\| \cdot \|_1)\rightarrow \mathbb{R}, f\mapsto M(f)$$
$$(L^1(\Omega)\bigcap B_r(L^{\infty}(\Omega)),\| \cdot \|_1)\rightarrow \mathbb{R}, f\mapsto \int^{\Lambda}_0 M(\lambda f)d\lambda $$
是上半连续的.

\end{lemma}
\end{frame}

\begin{frame}


\frametitle{几个引理}
\begin{proof}
$$\Phi'=const \cdot \prod_{j=1}^{3}{(1-\frac{z}{z_j})}^{-\frac{2}{3}}$$

$$\gamma_{f_n}\circ \Phi(0)=\frac{1}{2\pi}\int^{2\pi}_0 \gamma_{f_n}\circ \Phi(e^{i\theta})d\theta$$

因为$\gamma_{f_n}(\Phi(0))\rightarrow \gamma_f(\Phi(0)) \ n\to\infty$,所以
$$frac{1}{2\pi}\int^{2\pi}_0[\gamma_{f_n}\circ \Phi(e^{i\theta})-\gamma_{f}\circ \Phi(e^{i\theta})]d\theta \to 0$$
由控制收敛定理,沿着$T$的边$J,K$的积分趋近于$0$.
因此
$$\int_I[\gamma_{f_n}(E)-\gamma_f(E)]g(E)dE\to 0$$

\end{proof}
\end{frame}

\begin{frame}


\frametitle{几个引理}
\begin{proof}
由Lyapunov指数的对$f$上半连续性以及控制收敛定理,
$$\int_I max\{ \gamma_{f_n}(E)-\gamma_f(E),0 \}g(E)dE\to 0,$$

选择$\delta>0$使得集合$X=\{E \in I:\gamma_f(E)<\delta \}$的测度小于$M(f)+\frac{\varepsilon}{4}$.那么
$$\int_{I\backslash X}min\{\gamma_{f_n}(E)-\gamma_f(E),0\}dE\to 0.$$

这说明对$n \geq n_0$,存在测度小于$\frac{\varepsilon}{4}$的集合$Y_n$使得$\forall E\in I\backslash (X\bigcup Y_n)$有$\gamma_{f_n}(E) \geq \frac{\delta}{2}$.就有$\limsup M(f_n) \leq M(f)+\frac{\varepsilon}{2}$,这与3矛盾.

\end{proof}
\end{frame}

\begin{frame}
\begin{lemma}
$L^{\infty}(\Omega)$中存在稠密集$Z$使得$\forall s \in Z$满足
\begin{enumerate}
\item $s$只取有限个值
\item $s(T^n(\omega))$,对几乎所有的$\omega$不是周期的
\end{enumerate}

\end{lemma}
\end{frame}

\begin{frame}
\frametitle{主要定理}
\begin{theorem}
存在$C(\omega)$的剩余集,使得$\sigma_{ac}(f)=\phi$
且对几乎所有$\lambda>0$都有$\sigma_{ac}(\lambda f)=\phi$

\end{theorem}
\begin{proof}
$$M_\delta=\{ f \in C(\Omega):M(f)<\delta \}$$
由引理一,$M_\delta$是开的;由引理三,$M_\delta$是稠密的.
$$\{ f \in C(\Omega):\sigma(f)=\phi \}=\{ f \in C{\Omega):M(f)=0}=\bigcap_{\delta >0} M_\delta$$
是剩余集.
第二个断言的证明类似.
\end{proof}
\end{frame}


\end{document} 