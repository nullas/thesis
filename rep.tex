% !Mode:: "TeX:UTF-8"
\documentclass[compress,red]{beamer}
\usepackage{indentfirst,fancyhdr,graphicx}
\usepackage[SlantFont,BoldFont]{xeCJK}
\usepackage{mathrsfs}
\usepackage{amsmath}
\usepackage{amsfonts}
\usepackage{amssymb}
\usepackage{amsthm}
\usepackage{enumerate}
\usetheme{Frankfurt} % Beamer theme v 3.0
%\usecolortheme{lily}
%\beamertemplateshadingbackground{white!15}{yellow!50}

%\useinnertheme{rounded}


\setCJKmainfont[BoldFont={Adobe Heiti Std}, ItalicFont={Adobe Kaiti Std}]{Adobe Song Std}
\setCJKsansfont{Adobe Heiti Std}
\setCJKmonofont{Adobe Fangsong Std}
\setCJKfamilyfont{Microsoft YaHei}{Microsoft YaHei}
\setCJKfamilyfont{Adobe Song Std}{Adobe Song Std}
\setCJKfamilyfont{Adobe Heiti Std}{Adobe Heiti Std}
\setCJKfamilyfont{Adobe Kaiti Std}{Adobe Kaiti Std}
\setCJKfamilyfont{Adobe Fangsong Std}{Adobe Fangsong Std}
\setmainfont{Times New Roman}
\setsansfont[BoldFont={Courier New Bold}]{Courier New}
\setmonofont[BoldFont={Arial:style=Bold}]{Arial}


\begin{document}


\title{唯一遍历与一致次可加遍历定理}
\author{赵鑫}
\institute{中国科学技术大学数学科学学院数学系}
\date{\today}



\frame{\titlepage}





\begin{frame}
\frametitle{唯一遍历与一致次可加遍历定理}
本文主要研究了符号系统上的一致可加和次可加遍历定理.

\begin{itemize}
\item 本文首先介绍了一般动力系统上的不变测度和唯一遍历性.\pause

\item 给出了符号系统上极小性和唯一遍历的刻画.\pause

\item 本文最后证明了极小系统上一致可加和次可加遍历定理的充要条件.
\end{itemize}
\end{frame}
\section{不变测度与唯一遍历}
\subsection{保测变换}
\begin{frame}
\frametitle{保测变换}
\begin{definition}
设$(X_1,\mathscr{B}_1,m_1),(X_2,\mathscr{B}_2,m_2)$是概率空间.
\begin{itemize}
\item<1-> 变换$T:X_1 \rightarrow X_2$称为可测变换,如果$T^{-1}(\mathscr{B}_2) \subseteq \mathscr{B}_1$.(即\
$B_2 \in \mathscr{B}_2\Rightarrow T^{-1}(B_2) \in \mathscr{B}_1$)
\item<2-> 变换$T:X_1 \rightarrow X_2$是保测变换,如果$T$是可测的,且
$$m_1(T^{-1}(B_2))=m_2(B_2), \forall B_2 \in \mathscr{B}_2$$
\item<3-> 特别地,我们把概率空间$(X,\mathscr{B},\mu)$以及其上的保测自变换$T:X\rightarrow X$,我们称四元组$(X,\mathscr{B},\mu,T)$为保测系统.
\end{itemize}
\end{definition}
\end{frame}



\subsection{遍历性}
\begin{frame}
\frametitle{遍历}
\begin{definition}
设$(X,\mathscr{B},m,T)$是一个保测系统.$T$是遍历的,如果$B\in\mathscr{B} $满足$T^{-1}(B)=B$,那么$m(B)=0$或者$m(B)=1$.
\end{definition}
\end{frame}




\subsection{极小性}
\begin{frame}
\frametitle{极小性}
\begin{definition}
动力系统$(X,T)$称为极小的,如果他不真包含任何闭不变子空间.

\end{definition}
\end{frame}

\begin{frame}
\frametitle{遍历极小系统的例子}
\begin{example}[$S^1$的无理旋转]
设$X=S^1=\mathbb{R}/\mathbb{Z}$为一个圆周.$Tx=x+a$,其中$a$是一个无理数,$\mu$为$\mathbb{R}$上诱导的Lebesgue测度,则$(S^1,\mathscr{B}(S^1),\mu,T)$是遍历的极小系统.
\end{example}
\end{frame}





\section{不变测度}
\subsection{不变测度的存在性}
\begin{frame}
\frametitle{引子}
\begin{itemize}
\item<1-> 对于任何一个动力系统$(X,T)$,它本身并没有测度的结构.
\item<2-> 但是其本身有自然的由开集生成的$\sigma$代数.
\item<3-> 但是我们是否可以构造出一个Borel测度$\mu$并且是不变测度?
\item<4-> 如果我们能够做到这一个点,就可以把遍历性推广到任意一个动力系统上,继而运用于拓扑动力系统中.
\item<5-> 这样的测度是存在.

\end{itemize}

\end{frame}

\begin{frame}
\frametitle{两个引理}
\begin{theorem}[Riesz's Representation Theorem]

设$X$是紧致的度量空间.$I:C(X)\rightarrow \mathbb{C}$为线性的正算子,且$I(1)=1$,那么存在$u\in \mathcal{M}(X)$使得$$\int \!f\,\mathrm{d}u=I(f).$$

\end{theorem}
\begin{lemma}
\label{l}
$\mathcal{M}(X)$是动力系统$(X,T)$上的一个概率测度的全体.那么$u=v\in \mathcal{M}(X)$当且仅当$\int\!f \, \mathrm{d}u=\int f\, \mathrm{d}v,\forall f\in C(X)$.
概率测度$u$是不变的当且仅当$\int\!f \, \mathrm{d}u=\int\! U_Tf\, \mathrm{d}v,\forall f\in C(X)$
\end{lemma}
\end{frame}

\begin{frame}
对$\mathcal{M}(X)$我们可以给与弱$^*$拓扑,即$\mu_n\rightarrow \mu$当且仅当$\int\!f \, \mathrm{d}\mu_n\rightarrow \int\!f \, \mathrm{d}\mu,\forall f \in C(X)$.
在这个拓扑下,
\begin{theorem}
$\mathcal{M}(X)$是紧致的.
\end{theorem}

\end{frame}

\begin{frame}
\frametitle{存在不变测度}
\begin{theorem}[Krylov-Bogolioubov's Theorem]
对任意动力系统$(X,T)$,存在不变测度$\mu\in\mathcal{M}(X)$.
\end{theorem}
\end{frame}

\begin{frame}
\frametitle{遍历测度}
\begin{definition}
对动力系统$(X,T)$,如果$\mathcal{M}^e(X)$中仅有一个元素,我们称这个系统是唯一遍历的.

\medskip
这就是说所有的不变测度中,只有一个测度,使得$(X,T)$是一个遍历系统.
\pause

\medskip
如果这个系统还是极小的,那么就称为严格遍历.
\end{definition}

\end{frame}


\section{符号系统}
\subsection{符号系统}
\begin{frame}
\frametitle{符号系统的定义}
\begin{example}
字母表$A=\{1,\ldots,n\}$,\\ \pause 
词$w=(1,3,5,2,6\ldots)$,\\ \pause
变换定义为左移, \pause 
即$Tw=(3,5,2,6,\ldots)$.
\end{example}
\pause
\begin{definition}
称$(A^{\mathbb{Z}},T)$为转移.\pause

\medskip 
若$\Omega\subseteq A^{\mathbb{Z}}$在$T$下不变的闭空间,则$(\Omega,T)$称为子转移.
\end{definition}
\end{frame}

\begin{frame}
\frametitle{符号系统的极小性刻画}
在一般的动力系统中,下面的结果是基本的:
\begin{theorem}
$(X,T)$为极小的$\Longleftrightarrow$对任何非空开集$U$,存在有限子集$A\subset \mathbb{N}$,使得$\bigcup_{n\in A}T^{-n}U=X$
\end{theorem}
\pause
在符号系统中,极小性有如下的刻画:
\begin{theorem}
$(\Omega,T)$是极小的$\Longleftrightarrow$对每个$w\in \mathcal{W}$存在$R(w)>0$使得对%
每个$v\in \mathcal{W}$且$|v|\geq R(w)$,$w$都是$v$的一个因子.
\end{theorem}
其中$\mathcal{W}=\mathcal{W}(\Omega)$为所有$w\in \Omega$的有限子词全体.
\end{frame}

\begin{frame}
\frametitle{唯一遍历的刻画}
在符号系统中,极小性有如下的刻画:
\begin{theorem}
$(X,T)$是唯一遍历的\\$Longleftrightarrow$
$\forall f\in C(X)$,$\frac{1}{n}\sum_{k=0}^{n-1}f(T^kx)$逐点收敛到一个常值函数
\end{theorem}
\pause
在符号系统下的刻画:
\begin{theorem}
子转移$(\Omega,T)$是唯一遍历的\\$\Longleftrightarrow$ $\lim_{|w| \to \infty}{\#_v(w)\over |w |}<\infty, \forall v\in \mathcal{W}$
\end{theorem}
\end{frame}

\begin{frame}
\frametitle{两个引理}
\end{frame}
\begin{frame}
\frametitle{}
\end{frame}

\end{document} 